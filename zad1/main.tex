\documentclass{article}
\usepackage{graphicx} % Required for inserting images
\usepackage{listings}
\usepackage{color}
\usepackage[T1]{fontenc}
\usepackage[polish]{babel}
\usepackage[utf8]{inputenc}
\usepackage{hyperref}

\definecolor{dkgreen}{rgb}{0,0.6,0}
\definecolor{gray}{rgb}{0.5,0.5,0.5}
\definecolor{mauve}{rgb}{0.58,0,0.82}

\lstset{frame=tb,
  language=Go,
  aboveskip=3mm,
  belowskip=3mm,
  showstringspaces=false,
  columns=flexible,
  basicstyle={\small\ttfamily},
  numbers=none,
  numberstyle=\tiny\color{gray},
  keywordstyle=\color{blue},
  commentstyle=\color{dkgreen},
  stringstyle=\color{mauve},
  breaklines=true,
  breakatwhitespace=true,
  tabsize=3
}

\title{Go zadanie 1 - opracowanie}
\author{Kajetan Wiśniewski}
\date{Kwiecień 2024}

\begin{document}

\maketitle

\section{Wstęp}

Treść zadania:


Wygeneruj swój nick z 3 liter imienia i 3 liter nazwiska, następnie zamień go na ASCII i znajdź taką liczbę, której silnia zawiera w sobie wszystkie wartości numeryczne kodów ASCII z Twojego nicku. Będzie to Silna Liczba. Następnie wyznacz wartość 30 elementu Ciągu Fibonacciego, i oblicz liczbę wywołań każdego argumentu podczas tych obliczeń. Znajdź liczbę wywołań najbardziej zbliżoną do Silnej Liczby. Słaba liczba to argument dla którego wykonuje się ta liczba wywołań. Jeżeli nie rozumiesz o co chodzi, przeczytaj wersję długą.

Zadanie podzieliłem na dwa pliki. Jeden plik functions.go zawiera funkcje niezbędne do wykonania zadania, oraz plik main.go który zawiera funkcję main wywołującą potrzebne do obliczenia silnej i słabej liczby funkcje. Zawinięte jest to w package main. w folderze znajduje się również plik go.mod.

\section{functions.go}

\textbf{funkcja factorial} wylicza silnie. Ponieważ mamy do czynienia z dużymi liczbami, to funkcja operuje na typie big int i za pomocą operatorów z tego package'u.
\begin{lstlisting}
func factorial(x *big.Int) *big.Int {
	n := big.NewInt(1)
	if x.Cmp(big.NewInt(0)) == 0 {
		return n
	}
	return n.Mul(x, factorial(n.Sub(x, n)))
}
\end{lstlisting}

funkcja \textbf{generateNickname} bierze string jako input oraz zwraca dwie wartości, string i error. Dopasowana jest tak aby akceptować tylko input określony w zadaniu i zwracać 6-znakowy nickname.

\newpage
\begin{lstlisting}
func generateNickname(input string) (string, error) {

    parts := strings.Split(input, " ")
    if len(parts) != 2 {
        return "", fmt.Errorf("input powinien zawierac imie i nazwisko rozdzielone spacja")
    }
    name := strings.ToLower(parts[0])
    surname := strings.ToLower(parts[1])
    var nickname string
    if len(name) >= 3 && len(surname) >= 3 {
        nickname = name[:3] + surname[:3]
    } else {
        return "", fmt.Errorf("imie i nazwisko, oba, powinny miec przynajmniej 3 znaki dlugosci")
    }
    return nickname, nil
}
\end{lstlisting}

funkcja \textbf{nicknameToASCII} to prosta funkcja konwertująca wygenerowany nickname na 6-elementową tablicę zawierającą kody ASCII dla danego nickname'u.

funkcja \textbf{containsSubNumber} to prosta funkcją zamieniająca liczby typu big int oraz int na string i sprawdzająca czy w dużej liczbie mieści się podciąg.

funkcja \textbf{containsAllDigits} to funkcja boolowska wykorzystująca containsSubNumber sprawdzająca czy dany wynik silni zawiera wszystkie kody ASCII z 6-elementowej tablicy. 

funkcja \textbf{fibonacciWithCalls} to funkcja sprawdzająca kolejne wyrazy ciągu fibonnaciego wykorzystującą typ map[int]int do zapisywania liczby wywołań, gdzie kluczem jest kolejny wyraz ciągu, a wartością liczba wywołań (czyli kolejne wyrazy ciągu fibonnaciego). callCount[n-2] odpowiada na przesunięcie wyrazów które występuje bez tego warunku (np. wyraz 20 miał wartość wyrazu 18tego).

\begin{lstlisting}
    func fibonacciWithCalls(n int, callCount map[int]int) int {
	if n < 0 {
		callCount[n] = 1
		return 1
	}
	callCount[n-2]++
	return fibonacciWithCalls(n-1, callCount) + fibonacciWithCalls(n-2, callCount)
}
\end{lstlisting}

funkcja \textbf{abs} to po prostu funkcja zwracająca wartośc absolutną z wartości int i zwracająca też wartość int. math.Abs zwraca typ float64.

\section{main.go}

\textbf{Krok 1}

Pobranie od użytkownicka imienia oraz nazwiska, a następnie wygenerowanie nicku i przekonwertowanie go na ASCII. ta część odbywa się w nieskończonej pętli, aż do momentu gdy użytkownik wpisze poprawnie swoje imię i nazwisko. w przeciwnym razie wyskoczy błąd i zostanie poproszony o ponowne podanie danych.

\begin{lstlisting}
    func main() {

	var generatedNickname string
	var err error
	var nicknameInASCII [6]int

	for {
		fmt.Println("wpisz swoje imie i nazwisko rozdzielone spacja aby wygenerowac nickname:")
		reader := bufio.NewReader(os.Stdin)
		input, _ := reader.ReadString('\n')
		input = strings.TrimSpace(input)
		generatedNickname, err = generateNickname(input)
		if err == nil {
			nicknameInASCII = nicknameToASCII(generatedNickname)
			break
		}
		fmt.Println("Error:", err)
	}
\end{lstlisting}

\textbf{Krok 2}

Wyliczenie silnej liczby. Wykorzystywane są dwa iteratory, ponieważ później w programie mocno gmatwały się typy big int, int64 (wbudowana konwersja z package'u big jest z typu big int na int64) oraz int. Prosciej było zastosować drugi iterator i pomimo, że oba mają dokładnie te samą wartość liczbową to ich użycie w programie różni się. n, big.int używane jest do wyliczania silni, natomiast strongNumber do porównywania słabych liczb.
\begin{lstlisting}
    n := big.NewInt(1)
	strongNumber := 1
	for {
		factorialResult := factorial(n)
		if containsAllDigits(factorialResult, nicknameInASCII) {
			break
		}
		n.Add(n, big.NewInt(1))
		strongNumber++
	}
\end{lstlisting}

\newpage
\textbf{Krok 3}

Wyliczenie słabej liczby i prezentacja wyników. Do wyliczenia słabej liczby używana jest map callCount zbierajcy wyniki funkcji fibonacciWithCalls. w pętli porównywane są różnice kolejnych wartości z tego map z silną liczbą aby wyliczyć najmniejszą różnicę, a zatem słabą liczbę dla danego użytkownika

\begin{lstlisting}
    callCount := make(map[int]int)
	for i := 30; i >= 1; i-- {
		fibonacciWithCalls(i, callCount)
	}
	minDiff := strongNumber
	weakNumber := 0
	weakNumber2 := 0
	for i := 0; i <= 30; i++ {
		diff := abs(callCount[i]+1 - strongNumber)
		if diff < minDiff {
			minDiff = diff
			weakNumber = i
			weakNumber2 = callCount[i] + 1
		}
	}
	fmt.Printf("Silna liczba dla %s to %d a Slaba liczba to %d [dla %d wywolan]:\n", generatedNickname, strongNumber, weakNumber, weakNumber2)
}    
\end{lstlisting}
\begin{lstlisting}
    go run .
    wpisz swoje imie i nazwisko rozdzielone spacja aby wygenerowac nickname:
    Kajetan Wisniewski
    Silna liczba dla kajwis to 297 a Slaba liczba to 18 [dla 233 wywolan]
\end{lstlisting}

Gdyby chcieć wyliczyć fib(297) prawdopodobnie zajęło by to sporo czasu.. nie wiem czy bym się kiedykolwiek doczekał, mimo, że mam całkiem mocny komputer. Tymbardziej utwierdza mnie w tym w wniosku sprawdzenie wartości ciągu aż do 300 ze strony \href{https://r-knott.surrey.ac.uk/Fibonacci/fibtable.html}{Liczby Fibonnaciego}. Gdyby wsadzić te wartości jak 297 i 18 do funkcji ackermanna to zakładam, że czarne dziury zdążyłyby wyzionąć ducha a ta dalej by się liczyła.
\end{document}
